\section[The cooking]{USB stick preparation}
\subsection{Ingredients}
\begin{frame}{USB recipe, ingredients}
  \begin{description}[short]
  \item[Distribution] Ubuntu based host OS.
    \begin{itemize}
    \item Up to date kernels, more chance to use with new laptops.
    \item Otherwise quite stable
    \item Start with \Code{ubuntu-mini-remix-14.04.1-amd64.iso} from
      \url{http://www.ubuntu-mini-remix.org/}
    \item Use lightweight desktop like XFCE4.
    \item Current version 14.04.5 LTS, kernel 4.4 (same as 16.04 LTS).
    \end{itemize}
  \item[Apps]  Mix and match whatever you need. Most of the time with a
    simple \Code{apt-get install} sometimes special PPAs (e.g. Java 8).
  \item[Home grown packages] Maybe some \Okis{deb} [re]packaging required:
    NetBeans, Glassfish etc.\\
    Packaging is simple: Pickup an installed version of the app, add some meta
    info and checksum the lot. See script sample.
  \end{description}
\end{frame}

\subsection{Tools}
\begin{frame}{USB recipe, Tools}
  \begin{description}[short]
  \item[Host] computer running same OS version. This eases work.\\
    Tweak the max number of boot devices (add a parameter to grub command line).
  \item[Customizer] Simple but adequate. Version 3.2.3.
    \begin{itemize}
    \item Needs some tweaks to get it running. (Gambas 3 vs 2 on U 14.04).
    \item \url{https://github.com/clearkimura/Customizer}.
    \item Revived. They seem to be moving to python.
    \end{itemize}
  \item[Configure] Tip: configure customizer to use something
    different then the default working dir (/home), e.g
    /home/U14.04. This allows you to have multiple versions more easily.
  \item[Startup Disk Creator] (\Code{/usr/bin/usb-creator-gtk}),
    standard in Ubuntu.
  \item[DD] Good old \Code{dd}.
  \item[Cook] A good Linux Cook\InlineSmiley. Some knowledge is indispensable.
  \end{description}
\end{frame}

\subsection[Stir well]{Stir well}
\begin{frame}{Preparing the dough}
  \begin{columns}
    \begin{column}[c]{.6\textwidth}
      \begin{itemize}
      \item Prepare basic stick.
        \begin{itemize}
        \item Load iso file with Customizer. This will unpack iso into
          \Code{/home/U14.04/FileSystem} and live system helpers into \Code{/home/U14.04/ISO}
        \item Open \Okis{terminal} in customizer. This does 'apt-get
          update' automatically. 
        \end{itemize}
      \end{itemize}
    \end{column}
    \begin{column}[c]{.4\textwidth}
      \includegraphics[width=\textwidth]{figures/customizer.png}
    \end{column}
  \end{columns}
  \begin{itemize}
  \item In the terminal you have all apt-tools. You are in a \Code{chrooted} environment
  \item Install (apt-get) whatever you want, e.g. desktop and window manager
    (e.g. XFCE4) and any apps you like.
  \item Install local \Code{.deb} packages with 'install deb
    package'.
  \item Rebuild ISO which creates a new iso file in \Code{/home/U14.04}.
  \item Rename the iso, e.g. add a date to it. This ISO is also a
    snapshot, making it easy to revert your steps.
  \end{itemize}
\end{frame}

\begin{frame}{Prepare a stick.}
  \begin{itemize}
  \item Add exam user account and settings (e.g. 'empty password').
  \item Add correctors account if required.
  \item Add everything you need in the next steps to the
    \Code{/home/FileSystem}. That will land in the ISO and stick too.\\
    Example: JDK documentation for on stick browsing during exam.
  \item Bake one stick first. Simply use \textbf{Startup Disk Creator} from the
    Host menu. Allow for some work space. 1GB (the default) is
    typically just fine.
  \item Test it.
  \item If satisfying, continue to the next step, otherwise do some more
    tweaking in customizer as required.
  \item You could provide the ISO to the students, so they can
    practice with the USB and IDE on their device. I offered to
    create (batches) of sticks, owned by the students.
  \end{itemize}
\end{frame}

\subsection[EXAM proof]{Making it exam proof}
\begin{frame}{Make it exam proof}
  \begin{itemize}
  \item Remove kernel modules and drivers not needed.
    \begin{itemize}
    \item Network\footnote{You can leave wired network in, if you want
      a network based exam}, wifi, bluetooth, local hard drives.
    \end{itemize}
  \item Remember to do a \Code{depmod} -a inside the
    \textbf{customizer-terminal}. Otherwise your stick will not boot.
  \item Make a new ISO. Rename it properly.
  \item Test again of course.
  \end{itemize}
\end{frame}

\lstset{basicstyle={\scriptsize}}
\subsection[Production]{Prime for Production}
\begin{frame}{Prepare exam workspace}
  \begin{itemize}
  \item The live system auto-mounts two rw file systems, casper-rw and
    home-rw. Casper-rw is a file system inside a file and is used for
    any modifications to the base system. Think the /var file system.
    home-rw contains the users home  directories. Both are created in
    the prepareSticks script
    \begin{itemize}
    \item casper-rw is 1GB large.
    \item home-rw in an ext2 formatted partition sized for the
      remainder of the disk.
    \end{itemize}
  \item Bake a new stick and boot from it on exam machine to \textit{configure} it. This
    will create the exam user space under \Code{/home/exam}.
  \item Tweak settings and such (UI menu to make it less confusing,
    remove unneeded desktop icons), make browser bookmarks to local
    documentation, e.g. to JDK doc.
  \end{itemize}
\end{frame}

\begin{frame}{Prepare exam workspace, continued}
  \begin{itemize}
  \item Shut down exam test machine.
  \item Mount the primed stick with 
    \lstinputlisting[language=sh]{code/scripts/casper-simple-example}
  \item Copy  the primed \Okis{exam work space} date from
    \lstinline[language=sh,basicstyle={\normalsize},morekeywords={exam,$HFS}]{$HFS/exam/exam} %$
    to some place safe 
    for later use (stick cleanup). Use e.g. \Code{tar czf} or
    zip. Result: \Code{skell.tgz}
\end{itemize}
\end{frame}

\subsection[Baking]{Baking Sticks in Batches}
\begin{frame}[shrink]{Stir, bake at 5 Volts, 21 per batch}
  \begin{itemize}
  \item (sticky) Label your sticks with unique IDs, if you not already did so.
  \item Now connect your USB hubs and
  \item Insert your sticks, ensuring the host sees it. (as in: wait a
    bit between sticks).
  \item Insert in label order. In this phase it ensures proper matching of stick label and
    stick disk-label (EXAM100, EXAM101,...).
  \item start the \Code{prepareSticks} script with the number of the
    first stick. The script will eneumerate the sticks by itself and
    number the sticks in insertion order.
    This writes and labels all 21 sticks (the c parameter) in \textit{parallel},
    which takes about 120 seconds on my machine.
  \item The scripts writes to the mounted sticks in parallel. The
    aggregated write speed of all stics does not exceed the bus speed
    on a 7 port hub.
  \end{itemize}
\end{frame}

\begin{frame}{The topping, Adding the tasks}
  \begin{itemize}
  \item In our setup, the students receive a stick, personalized, with
    the appropriate NetBeans project pre installed in
    \Code{/exam/exam/Desktop/examproject} or similar.
  \item We prepare the exam from a solution project
    \begin{itemize}
    \item adding corrector tags and TODO tags (manual labor).\\
      Here the corrector's workbench NetBeans plugin might be handy.
    \item strip the solution
    \item Make personalized copies with one copy per stick
      (e.g. EXAM101), this is the \Okis{local sandbox}.
    \item Put them in a repository, one per student, if you so require.
    \end{itemize}

  \item Insert the sticks in the hubs, connected, the disk label now identifies the sticks.
  \item Use the \Code{primeSticks} command, no parameters; it can find
    how many sticks are mounted.
    \begin{itemize}
    \item Installs fresh copy of the exam work space. (Dropping and
      earlier content)
    \item rsyncs local sandboxes into exam work space.
    \end{itemize}
  \end{itemize}
\end{frame}

\begin{frame}{The proof is in the eating}
  \begin{itemize}
  \item Take the exam. Steam the students. 3 hours and they are well done.
  \item Collect the sticks.
  \item \textit{Harvest} the student projects.
    \begin{itemize}
    \item Insert the sticks into the hubs
    \item use the command \Code{harvestSticks} to copy both sandbox
      and stick local repo back to the preparation host.
      \begin{itemize}
      \item Takes about 20 seconds for 21 sticks.\\
      \item Inserting and removing sticks takes about the same time.
    \end{itemize}
  \item Commit the work into the repository.
  \end{itemize}
  \item Make the exam work available to the correctors.
    \begin{itemize}
    \item Now using simple PHP based web app, to allow parallel
      corrections by multiple correctors. DEMO.
    \end{itemize}
  \end{itemize}
\end{frame}

